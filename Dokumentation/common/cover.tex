\begin{titlepage}
\newgeometry{top=2cm, bottom=2cm, left=2cm, right=2cm}

%\definecolor{color_30879}{rgb}{0,0.12549,0.376471}

\newcommand{\stitle}{\fontsize{12}{1}\selectfont\color{NAKblue}}
\newcommand{\btitle}{\fontsize{14}{1}\selectfont\color{NAKblue}}




\centering
    %% Logo oben
    \includegraphics[width=0.75\textwidth]{images/nordakademie_logo.png}
    \vspace{10mm}

    \begin{spacing}{2}
        {\textbf{\fontsize{16}{1}\selectfont\color{NAKblue} Wie lassen sich ML- und LLM-Systeme angesichts Angriffsformen in den Bereichen Inferenz, Training und RAG wirksam absichern, und wie ergänzen regulatorische Governance-Mechanismen die Grenzen technischer Verteidigungsstrategien?}}\par
        \vspace{1cm}
        \textbf{{\btitle Bearbeitungszeit der Hausarbeit:}}\par
        {\stitle Sonntag, den 30.11.2025 bis Sonntag, den 21.12.2025}\par
        \vspace{1cm}
        \textbf{{\btitle Erarbeitet von:}}\par
        {\btitle Gruppe A}\par
        {\stitle Anastasia Klat, Matrikelnummer: 13275}\par
        {\stitle Felix Pommerening, Matrikelnummer: 13243}\par
        {\stitle Julian Falk, Matrikelnummer: 13498}\par
        {\stitle Studiengang: Technische Informatik}\par
        {\stitle Zenturie: T23a}\par
        \vspace{1cm}
        \textbf{{\btitle WPM - Einführung in die KI }}\par
        {\stitle Dozenten und Prüfer:}\par
        {\stitle Prof. Dr. Arne Ewald}\par
        {\stitle Dr. Fereshta Yazdani}
    \end{spacing}

    
\renewcommand\thefootnote{}
\footnotetext{
  Zur besseren Lesbarkeit wird in dieser Hausarbeit das generische Maskulinum verwendet. Die in dieser Arbeit verwendeten Personenbezeichnungen beziehen sich - sofern nicht anders kenntlich gemacht - auf alle Geschlechter.
}

\end{titlepage}