\section{Einleitung}
\subsection{Zweck des Geräts}\label{ZweckDesGeraets}
Das vorliegende Gerät \textit{"Der Pinkler"} wurde im Rahmen eines Projektes, in Kooperation mit der Technischen Universität Bergakademie Freiberg, in dem Modul Eingebettete Systeme der NORDAKADEMIE, von den Studenten Leyna Haardt, Anastasia Klat, Max Meteling, Nick Schröder und Malte Wendeler entwickelt. Der Pinkler dient der präzisen und automatisierten Verdünnung von Proben in einem Chemielabor.
\subsection{Zielgruppe der Anleitung}
Diese Anleitung richtet sich an Laborpersonal und Techniker, die das Gerät bedienen, warten und gegebenenfalls reparieren müssen. Vorkenntnisse im Umgang mit Laborgeräten und grundlegende technische Kenntnisse sind von Vorteil.
\subsection{Kurzbeschreibung des Systems}
Das System besteht aus mehreren Hauptkomponenten, darunter ein mechanisches Gestell mit Achsen für die Bewegung eines Spritzkopfes, mehrere Pumpen zur Flüssigkeitsförderung, eine zentrale Steuerungseinheit zur Automatisierung der Prozesse, sowie eine grafische Oberfläche zur Bedienung des Geräts. Dadurch wird eine präzise, wiederholbare und zuverlässige Verdünnung von Proben ermöglicht.
