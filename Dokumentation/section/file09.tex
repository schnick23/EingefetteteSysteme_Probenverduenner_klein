\section{Technische Dokumentation (Anhang)}
\subsection{Schaltplan}
\subsection{3D-Druck-Pläne}
\subsection{Stückliste} \label{Materialien}
In der folgenden Tabelle sind alle im Probenverdünner verwendeten Materialien und deren Eigenschaften aufgelistet. 

\begin{table}[H]
    \centering
    \begin{tabular}{|l|l|l|}
        \hline
        \textbf{Bauteil} & \textbf{Material} & \textbf{Bemerkungen} \\
        \hline
        Aluprofil & Aluminium & 9x440mm; \\
        \hline
        Spritzen & Polypropylen (PP) & Beständig gegen viele Chemikalien \\
        \hline
        3D-gedruckte Teile & PLA (Polylactid) & Eingeschränkt beständig \\
        \hline
        Gehäuse & Aluminium & Beständig gegen Korrosion \\
        \hline
        Dichtungen & Nitrilkautschuk (NBR) & Beständig gegen Öle und Fette \\
        \hline
    \end{tabular}
    \caption{Materialien und deren chemische Beständigkeit im Probenverdünner}
    \label{tab:materialien}

\end{table}