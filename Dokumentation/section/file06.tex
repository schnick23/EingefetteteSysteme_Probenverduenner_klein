\section{Funktionsweise}
\subsection{Grundprinzip der Probenverdünnung}
Erklärung des Verdünnungsprozesses\\
Was passiert grundsätzlich im \\
(Berechnung der Verdünnungsverhältnisse, Mischvorgang)\\
\subsection{Ablauf eines Verdünnungsvorgangs}
Nach Eingabe der Daten, Platzierung der Proben ...\\
Schritt-für-Schritt Beschreibung des Ablaufs\\
Startposition anfahren\\
Initiale Reinigung\\
Verdünnungsschritt\\
Reinigungsschritte zwischendurch\\
Wiederholung bis gewünschte Menge erreicht ist\\
Endreinigung\\
Abschlussreinigung\\
Leeren der Schläuche und Spritzen\\
Abdeckung Ja/Nein?\\
Wenn Ja dann fertig\\
Wenn Nein können Proben manuell entnommen werden\\
Entnehmen der Schockflasche, falls nötig --> dabei auf mögliche Reste in den Schläuchen achten\\
\subsection{Fehlerfälle}\label{Fehlerfaelle}
Typische Fehler und deren Auswirkungen auf den Ablauf\\
Beispiel: Probenbehälter leer, Achsenblockade\\
Reaktion des Systems auf Fehler\\
Hinweise zur Fehlerbehebung\\