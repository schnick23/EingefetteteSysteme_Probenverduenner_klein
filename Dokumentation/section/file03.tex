\section{Technische Spezifikationen}\label{TechnischeSpezifikationen}
\subsection{Systemübersicht}\label{Systemuebersicht}
Die folgenden Kapitel bieten einen Überblick über die Hauptkomponenten der Maschine, deren Funktion und ihre technischen Möglichkeiten.
\subsubsection{Steuerungseinheit}
Die gesamte Technik wird mithilfe von zwei Netzteilen, ein 12-Volt- und ein 24-Volt-Netzteil, mit Strom versorgt. Die werden wiederum über die Steckdose bestromt. An das 12 Volt Gerät ist wiederum ein Spannungswandler, welcher die passende Spannung von 5 Volt an den Raspberry Pi (Raspberry Pi 4 Model B) weiterleitet. Dieser fungiert als zentrale Steuerungseinheit. In ihm steckt eine SD Karte, auf der die Software gespeichert ist, welche die grafische Oberfläche erzeugt, sowie die Steuerungslogik, mit welcher die anderen Komponenten angesteuert werden. Eine dieser anderen Komponenten ist das Acht-Kanal-Relais, über welches die Pumpen einzeln oder zusammen angesteuert und auch hin und zurück pumpen können. Weiterhin sind an dem Pi die Driver der Stepper-Motoren und der Linearmotor selber angeschlossen.\newline
\textbf{Hier sollte noch ein Bild eingefügt oder referenziert werden auf dem die Unterseite beschriftet zu sehen ist}
\subsubsection{Linearantrieb}
Im vorherigen Kapitel wurde ein Linearmotor erwähnt, dieser ist fest an dem Linearantrieb angebracht und sorgt dafür das der Spritzkopf in X-Richtung, entlang der großen Gewindestange, bewegt werden kann. Der Linearantrieb fährt mit einer Geschwindigkeit von \textit{hier die Geschwindigkeit noch angeben}, und erreicht die gesetzten Positionen auf den Zehntelmilimeter genau. \newline
\textbf{Hier sollte noch ein Bild eingefügt oder referenziert werden auf dem der Linearantrieb und ihr Motor zu sehen ist.}
\subsubsection{Pumpen}
Insgesamt werden in dieser Maschine fünf Pumpen verwendet, welche das Verdünnungsmittel mit einem maximalen Fehler von 0,5 \% auf 10ml aufziehen. Wie gesagt werden diese über ein Relais angesteuert, sodass sie gleichzeitig pumpen. Allerdings gibt es minimale Zeitversetzungen zwischen den einzelnen Pumpen, um die gleiche Menge, somit wird jedoch die bestmögliche Genauigkeit erzeugt. An die Pumpen sind jeweils zwei Schläuche angeschlossen also insgesamt zehn. Fünf der Schläuche müssen vor Start des Programms in die Schockflasche gelegt werden, welche zu jeder Zeit einen Mindestfüllstand von 50 ml haben muss. Die anderen fünf sind an dem Spritzkopf befestigt.\newline
\textbf{Hier sollte noch ein Bild eingefügt oder referenziert werden auf dem die Pumpen beschriftet zu sehen sind}
\subsubsection{Spritzenkopf}
Der Spritzenkopf ist an einem Aluprofil befestigt, welches zwischen zwei Linearführung befestigt ist. Er wird von dem Linearantrieb betrieben. An ihm sind dazu noch fünf Spritzen und Schläuche montiert, welche die Falcontubes befüllen. Die Spritzen werden über einen Steppermotor und einem 3D-gedrucktem Konstrukt gleichmäßg und gleichzeitig aufgezogen, sodass fünf Proben oder weniger gleichzeitig befüllt oder entnommen werden können. Die Spritzen arbeiten mit einem maximalen Fehler von 3\% auf 1 ml. Zuletzt ist an dem Spritzkopf noch eine Platte befestigt, welche am Ende einer Verdünnung die Falcontubes bedecken kann.\newline
\textbf{Hier sollte noch ein Bild eingefügt oder referenziert werden auf dem der Spritzkopf beschriftet zu sehen ist}
\subsubsection{Hubtisch}
Auf dem Hubtisch befinden sich der Falcon-Tube-Ständer, ein Reinigungsbehälter, sowie ein Abfallbehälter. Im Normalfall würde dieser manuell über ein Gewinde betätigt werden. In dem Probenverdünner wird jedoch die Hoch-Runter-Bewegung über ein Stepper-Motor automatisiert und kann verschieden Positionen anfahren. Wie zuvor in \autoref{MechanischeSicherheitshinweise} erwähnt musste eine Expandierseil als Notlösung zur Unterstützung des Motors implementiert werden.  \newline
\textbf{Hier sollte noch ein Bild eingefügt oder referenziert werden auf dem der Hubtisch beschriftet zu sehen ist}
\subsubsection{Technische Möglichkeiten}\label{TechnischeMoeglichkeiten}
Der \textit{Name des Geräts} ist in der Lage 15 Proben in 23 Minuten zu befüllen. Der maximale Verdünnungsfaktor beträgt 1/1000. In diesem Fall lassen sich fünf Proben verdünnen. Bei einem Verdünnungsfaktor von unter 1/100 lassen sich zehn verdünnte Proben befüllen und bei einem Verdünnungsfaktor von unter 1/10, 15. Alle Verdünnungsverhältnisse, die nicht vom Probenverdünner umgesetzt werden können, werden durch die Software abgefangen.