\section{Erstinbetriebnahme des Probenverdünners}

\subsection{Anschluss und Start}

Für die erste Inbetriebnahme des Probenverdünners gehen Sie bitte wie folgt vor:

\begin{enumerate}
    \item Schließen Sie ein \textbf{LAN-Kabel} an den Probenverdünner an.
    \item Verbinden Sie das Gerät mit dem Stromnetz, indem Sie den \textbf{Kaltgerätestecker} einstecken.
    \item Schalten Sie den Probenverdünner über den \textbf{Netzschalter} ein.
\end{enumerate}
\begin{figure}[H] % r = rechts, l = links
    \centering
    \includegraphics[width=0.5\textwidth]{images/Stecker.png}
    \caption{Anschlüsse des Probenverdünners}
    \label{fig:anschluss}
\end{figure}

Das System startet anschließend automatisch. Nach einer Hochfahrzeit von ca. \textbf{30 Sekunden} sollte das Programm vollständig laufen und der Probenverdünner im Netzwerk erreichbar sein.

\subsection{Zugriff über das Netzwerk}

Sobald das Gerät gestartet ist, kann von einem Rechner im \textbf{gleichen Netzwerk} auf den Probenverdünner zugegriffen werden.  
Empfohlen wird zunächst ein Test der SSH-Verbindung.

\subsubsection{SSH-Verbindung herstellen}\label{SSH}

Öffnen Sie auf Ihrem Rechner ein Terminal und führen Sie folgenden Befehl aus:

\begin{verbatim}
ssh pi@pinkler.local
\end{verbatim}

Das Standardpasswort lautet:

\begin{verbatim}
pi
\end{verbatim}

Wenn die Verbindung erfolgreich hergestellt werden kann, ist das Gerät korrekt im Netzwerk eingebunden.

\subsection{Einrichtung einer WLAN-Verbindung}

Nach erfolgreicher SSH-Verbindung kann optional eine WLAN-Verbindung eingerichtet werden. Dies erfolgt über die Kommandozeile mit Hilfe von \texttt{nmcli}.

\subsubsection{Schritt 1: Neue WLAN-Verbindung anlegen}

\begin{verbatim}
sudo nmcli connection add type wifi ifname wlan0 con-name mywifi ssid "DEIN_WLAN"
\end{verbatim}

\subsubsection{Schritt 2: Sicherheit explizit festlegen (WPA2)}

\begin{verbatim}
sudo nmcli connection modify mywifi wifi-sec.key-mgmt wpa-psk
\end{verbatim}

\subsubsection{Schritt 3: WLAN-Passwort setzen}

\begin{verbatim}
sudo nmcli connection modify mywifi wifi-sec.psk "DEIN_PASSWORT"
\end{verbatim}

\subsubsection{Schritt 4: WLAN-Verbindung aktivieren}

\begin{verbatim}
sudo nmcli connection up mywifi
\end{verbatim}

\subsection{Fehlerbehebung bei Nichterreichbarkeit}

Sollte der Probenverdünner nicht per SSH erreichbar sein, wenden Sie sich bitte an die zuständige \textbf{lokale IT-Abteilung}.  
Diese kann mithilfe von Netzwerkwerkzeugen (z.\,B. \texttt{nmap} oder über die Netzwerkeinstellungen) die IP-Adresse des Geräts ermitteln und gegebenenfalls einen DNS-Eintrag setzen.

\subsection{Aufruf der grafischen Benutzeroberfläche}

Nach erfolgreicher Netzwerkeinbindung kann die grafische Benutzeroberfläche (GUI) im Webbrowser aufgerufen werden:

\begin{itemize}
    \item \texttt{http://pinkler.local:5001}
    \item oder alternativ \texttt{http://IP-ADRESSE:5001}
\end{itemize}

\subsection{Überprüfung des Systemdienstes}

Falls die GUI nicht erreichbar sein sollte, stellen Sie erneut eine SSH-Verbindung her und überprüfen Sie den Status des Systemdienstes:

\begin{verbatim}
sudo systemctl status probenverduenner.service
\end{verbatim}

Falls der Dienst nicht läuft, kann er manuell gestartet werden:

\begin{verbatim}
sudo systemctl start probenverduenner.service
\end{verbatim}

\subsection{Neuinstallation der Software}

Sollte der Dienst wider Erwarten weiterhin nicht funktionieren, führen Sie bitte folgende Befehle aus, um die Software neu zu installieren:

\begin{verbatim}
cd /home/pi/EingefetteteSysteme_Probenverduenner_klein/deploy

./install.sh
\end{verbatim}

Nach Abschluss der Installation starten Sie das Gerät neu.  
In der Regel sollte der Probenverdünner anschließend wieder ordnungsgemäß funktionieren. Sollte dies nicht der Fall sein, kontaktieren Sie bitte den Support.
