\section{Sicherheit}
\subsection{Allgemeine Sicherheitshinweise}
Der entwickelte Probenverdünner dient ausschließlich dem zuvor beschriebenen Zweck (\autoref{ZweckDesGeraets}) der Verdünnung von Proben. Um zu prüfen, ob der \textit{"Name des Geräts"} die Anforderungen der geplanten Verdünnung erfüllt, kann in Kapitel \autoref{TechnischeSpezifikationen} nachgeschaut werden. Zudem sollte das Gerät nur von eingewiesenem Personal, welches entweder, durch das Lesen von diesem Dokuments oder durch eine Einweisung von bereits geschultem Personal, sich mit dem System vertraut gemacht hat. Das Technik des Geräts ist komplex und empfindlich und sollte daher nicht verändert werden. Sollten allerdings Fehler auftreten können diese, wie in \autoref{Fehlerfaelle} beschrieben, behoben werden. Außer derjenige Fehler ist nicht in diesem Dokument behandelt, dann sollte sich an die Entwickler des Projektes gewendet werden(\textit{Noch eine Möglichkeit zur Kontaktaufnahme angeben und hier referenzieren}).
Es ist wichtig, dass die Bedienungsanleitung vor der Inbetriebnahme vollständig gelesen wird. Des Weiteren ist das Gerät, nach den in \autoref{WartungUndReinigung} beschriebenen Methoden zu überprüfen, warten und reinigen. Bei Schöden oder Fehlfunktionen ist das System sofort außer Betrieb zu nehmen.Betätigen Sie hierzu den Not-Aus-Knopf. Wie dieser zu bedienen ist folgt im nächsten Kapitel \nameref{ElektrischeSicherheitshinweise}.
 
- Nur für den vorgesehenen Zweck verwenden.\\
- Nur von geschultem Personal bedienen lassen.\\
- Vor der Inbetriebnahme Bedienungsanleitung vollständig lesen.\\
- Gerät nicht verändern oder umbauen.\\
- Gerät regelmäßig warten und überprüfen.\\
- Bei Schäden oder Fehlfunktionen Gerät sofort außer Betrieb 
- Bestimmungsgemäße Verwendung des Geräts erläutern
nehmen und Fachpersonal informieren.\\
\subsection{Elektrische Sicherheitshinweise}\label{ElektrischeSicherheitshinweise}
Der \text{"Name des Geräts"} wird über ein 230V Netzteil angeschlossen und mit Strom versorgt. Nahezu die komplette Elektronik des Geräts befindet sich auf der Unterseite der Bodenplatte (\textit{hier Bild der offenen Unterseite einfügen}). Deswegen sollte dieser vorsichitig auf einem stabilen, und noch wichtiger, trockenem Untergrund platziert werden. Es sollte außerdem zu jeder Zeit darauf geachtet werden, dass keine Flüssigkeiten verschüttet werden; sowohl über, als auch neben dem Gerät, da dieses ,zusätzlich zu der offenen Elektronik, auch Löcher für die Kabel einiger Geräte in dem Boden hat.\par
Sollte es dazu kommen, dass der Probenverdünner im laufenden Betrieb auf einen unerwarteten Fehler stößt oder beschädigt wird, ist sofort der Not-Aus-Schalter zu tätigen (\textit{hier Bild des Not-Aus-Schalters einfügen}). Nachdem dieser gedrückt wurde fährt sich sämtliche Elektronik des Geräts runter und alles bleibt dort stehen wo es ist. Um das Gerät nach dem Notfall wieder einzuschalten, muss der Not-Aus-Knopf einmal in die Richtung gedreht werden, die auf dem Knopf selber beschrieben ist. Danach sollte sich wie gewohnt die grafische Oberfläche öffnen. Dort findet man den Button \textbf{Reset}, durch welchen der Probenverdünner zurück an die Nullpositionen fährt. Bei Bedarf kann dann auch eine Reinigung initiiert werden oder Sie können die Proben dann entnehmen.

- Offene Elektronik auf Gerätunterseite\\
- Elektrisches Gerät --> Vorsicht im Umgang mit Flüssigkeiten.\\
- Not-Aus Schalter vorhanden --> Funktion und Position erläutern.\\
\subsection{Mechanische Sicherheitshinweise}\label{MechanischeSicherheitshinweise}
Aus Kostengründen mussten hinsichtlich der Mechanik an ein paar Stellen improvisiert werden. Dies führt zu wenigen Punkten die bei der Bedienung zu beachten sind. Beispielsweise wurden die Füße, sowie die Steckerhalter für Netzwerk und Strom 3D-gedruckt. Diese sind zwar stabil aber können bei unvorsichtiger Benutzung brechen. Sollte dies auftreten, können die nötigen Pläne, um diese erneut zu drucken, unter folgendem Link gefunden werden \textit{hier die 3D Drucke verlinken}. Es empfielt sich das Gerät nicht zu nutzen, solange die Teile noch nicht wieder ausgetauscht wurden. Leider ist der gewählte Motor, welcher den Hubtisch antreibt, nicht stark genug, um die Last der Proben zusammen mit den Widerständen, die durch das Gewinde und die Führstangen des Hubtisches entstehen, zu bewegen. Daher wurde ein Expandierseil mit Haken zur Unterstützung an dem Rahmen und dem Hubtisch befestigt.\par
Während des gesamten Durchlaufs darf nichts in dem \textit{Namen des Geräts} landen. Insbesondere die Linearführung des Spritzkopfes birgt eine Quetschgefahr. Zudem ist der Probenverdünner nicht für hohe Lasten geeignet. Das Gerät sollte nicht als Ablage oder Stütze genutzt werden.

- Hubtisch --> improvisierte Unterstützung\\
- Quetschgefahr --> während des Betriebs nicht in den Gefahrenbereich greifen oder Gegenstände platzieren (Haare, Hände etc.)\\
- Füße des Geräts könnten brechen --> Gerät nur auf ebenen, stabilen Untergrund stellen\\
- nicht dagegen lehnen oder abstützen\\
\textit{
- Position des Not-Aus Schalters beschreiben und Funktion erläutern.\\
- Funktionsweise des Not-Aus Schalters erklären (z.B. Stromzufuhr unterbrechen, Maschine sofort stoppen).\\
- Verhalten nach Betätigung des Not-Aus Schalters erläutern (z.B. Maschine erst wieder starten, wenn Gefahr beseitigt ist).\\
- Wiederinbetriebnahme nach Betätigung des Not-Aus Schalters beschreiben (z.B. Reset-Knopf drücken, Maschine neu starten).\\}
\subsection{Chemische Sicherheitshinweise}
Hinweise zum Betrieb im Laborumfeld geben.\\
- Material der Schläuche beachten\\
- Material der Spriten beachten\\
- Material der 3D-gedruckten Teile beachten\\
